\documentclass[9pt,a4paper]{article}

\usepackage{zed-csp,graphicx,color}%from
\begin{document}
\begin{titlepage}
  \begin{figure}[h]
  \centerline{\small MAKERERE 
  \includegraphics[width=0.1\textwidth]{muk_log} UNIVERSITY}
\end{figure}
\centerline{COLLEGE OF COMPUTING AND INFORMATIC SCIENCES}
\paragraph{•}
\centerline{DEPARTMENT OF COMPUTER SCIENCE\\}
\paragraph{•}

\centerline{COURSEWORK: RESEARCH METHODOLOGY(BIT 2207)\\}
\paragraph{•}

\centerline{LECTURER: MR.ERNEST MWEBAZE}
\paragraph{•}

\centerline{TOPIC\\}Email mark up, google developer product \\
\paragraph{•}
\centerline{COMPILED BY: \
 NAKUBULWA AGNES}
 \paragraph{•}
\centerline{STUDENT NUMBER :216017951}
\paragraph{•}
\centerline{REGISTRATION NUMBER:16/U/8895/PS}
\paragraph{•}
\end{titlepage}
\pagenumbering{roman}

\newpage
\pagenumbering{arabic}
\section{INTRODUCTION}
\cite{lee2003web}Email markup is a google developer tool that uses structured data in emails to work. Inbox and Gmail support both JSON-LD and Microdata and you can use either of them to markup information in email. This lets Google understand the fields and provide the user with relevant search results, actions, and cards. For example, if the email is about an event reservation, you might want to annotate the start time, venue, number of tickets, and all other information that defines the reservation.\cite{spellman2002design}
\section{}
Email Markup is a structured data format meant to enhance how certain kinds of emails are displayed. When properly registered and implemented, marked up email messages will show additional information to users in Google apps like Inbox or even Calendar. 
The goal of email markup is to allow people to take action on emails as quickly and simply as possible. For marketers, there are both pros and cons of this feature. In this post, we're going to look at the email markup options currently available, who can use it, and if it's worth.\cite{dhanjal2012markup}
\section{}
Email MarkUp highlights your emails with important information and actions in inbox by Gmail. Allows increase of user engagement with interactive buttons in one’s emails thus adding actions to emails. Emails with mark up appear in google search results when users search for tickets, flights and events-answers on search. Trigger confirmation cards so as to present cards to the user in the right place and time.\cite{sadoghi2013pooled}
Another consideration for using email markup is tracking. If you rely heavily on the ability to track email opens and clicks to trigger autoresponders and other marketing automation actions, you may not want to give your subscribers the option to bypass opening your email and clicking on your link.\cite{lee2002web}


\section{CONCLUSIONS.}

Email mark-up allows users to carry out actions directly in Gmail such as write a review, track a parcel, or view an order. Before you can use email markup, you must register with Google. Google will check to make sure you meet email sender quality guidelines, bulk sender guidelines, and action / schema quality guidelines. Following the above importances I strongly suggest for the inclusion of email markup In our daily mailing activities
\newpage
\bibliographystyle{IEEEtran}
\bibliography{References}
\end{document}